\documentclass[../revisedmain.tex]{subfiles}
\begin{document}
We can use the integral as an antiderivative! It is common to find the antiderivative with $\int_{c}^{x}$ for some constant $c$, so when finding an antiderivative with the integral, we drop writing the $c$ and the $x$:$\int$. The reason we can drop the bounds is because the lower bound will return a constant. We don't care about what that constant is, we can always find it later. \\\par Without bounds, we say that it is an \textbf{indefinite integral}. As seen at the beginning of \textsection 4, a function created by a derivative has an infinite amount of antiderivatives. Therefore, we write the resulting antiderivative as the indefinite integral$+C$, where $C$ is the \textbf{constant of integration}. $C$ is just the placeholder for some constant.\\\par For any funciton created by a derivative\footnote{I keep saying created by a derivative because there are some functions that do not have an antiderivative; they cannot be created with a derivative.}, the indefinite integral can calculate the antiderivative. For example, for $f'(x)$:$$f(x)=x^2+5x-1$$$$f'(x)=2x+5$$we can use the integral to reverse and find our original equation $f(x)$$$\int f'(x)dx$$$$\int 2x+5\,dx$$$$x^2+5x+C$$to which $f(x)$ is a valid solution when $C=-1$.
\end{document}