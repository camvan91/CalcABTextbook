\documentclass[../revisedmain.tex]{subfiles}
\begin{document}
One could try to find the integral the long summation way, however it is just simpler and time-reducing to memorize rules similar to the derivative rules.
\begin{enumerate}
	\item The integral of $f(x)+g(x)$ is equal to the integral of $f(x)$ + the integral of $g(x)$ evaluated from $a$ to $b$
	\item The integral of $c*f(x)$ for some constant $c$ is equal to c * the integral of $f(x)$ evaluated from $a$ to $b$
	\item The integral of $dx$ is $b-a$
	\item The integral of $x^n$ is $\displaystyle\frac{x^n+1}{n+1}$ evaluated from $a$ to $b$
	\item The integral of $x^{-1}$ is $\ln(x)$
	\item The integral of $e^u$ is $e^u$ evaluated from $a$ to $b$
	\item The integral of $\displaystyle\frac{1}{x}$ is $\ln(x)$ evaluated from $a$ to $b$
	\item The integral of $\sin(x)$ is $-\cos(x)$ evaluated from $a$ to $b$
	\item The integral of $\cos(x)$ is $\sin(x)$ evaluated from $a$ to $b$
	\item The integral of $f(x)$ from $a$ to $b$ is equal to the opposite (-) of the integral of $f(x)$ evaluated from $b$ to $a$
\end{enumerate}
A function $f(x)$ evaluated from $a$ to $b$ is the same as $f(b)-f(a)$. We usually represent it as $f(x)]_a^b$ \\The integral has some curious properties. However, it only returns a real number and not a function like the derivative because all it does is calculate the area.	The properties are eerily familiar though...\\
\end{document}