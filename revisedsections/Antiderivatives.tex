\documentclass[../revisedmain.tex]{subfiles}
\begin{document}
So far we have worked with derivatives. The derivative is a function that eats a function and spits out its rate of change. Because math is full of inverse operations (addition and subtraction, multiplication and division, etc.), shouldn't the derivative have an inverse? It is obvious now that the derivative of $x^2$ is $2x$, so the inverse of the derivative (or antiderivative) of $2x$ should be $x^2$. This statement isn't entirely true as $x^2$, $x^2-1$, and $x^2+\pi$ all share $2x$ as their derivative. Instead, we can say that the antiderivative of $2x$ is $x^2+C$ where $C$ is any constant. It should be noted that a function which is created by a derivative will have an infinite number of antiderivatives but only one derivative itself. Later in this section, you will read about ways to solve specific antiderivative problems.
\end{document}