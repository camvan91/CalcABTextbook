\documentclass[../revisedmain.tex]{subfiles}
\begin{document}
Because the integral is the antiderivative, the antiderivative of a function will return another function that can be used to find the area under the graph or net accumulation of the original function. This was touched upon in \textsection 4.6. In any case, the antiderivative finds many uses and can be used to calculate the average value of a function and also how much value the graph has underneath it. The most important takeaway here is (in many cases):$$\text{The Antiderivative} = \text{The Integral}$$	
\end{document}