\documentclass[../revisedmain.tex]{subfiles}
\begin{document}
\par The derivative has patterns, or ``forms'' as they will be called here. You are expected to know these forms backwards and forwards. Many textbooks include proofs for these but, as it is unnecessary, this text will not. The reader is strongly encouraged to prove these forms as exercises. Some of the most important forms that Calculus students are expected to know are below. The derivative of:
\begin{enumerate}
	\item $c$ is 0
	\item $u^n$ is $(nu^{n-1})\text{ } du$
	\item $cu$ is $c*(du)$
	\item $e^u$ is $e^u\text{ } du$
	\item ln($u$) is $\frac{1}{u}\text{ }du$
	\item sin($u$) is cos($u$) $du$
	\item cos($u$) is -sin($u$) $du$
\end{enumerate}
Where $u$ is some equation, $c$ is a constant, and $du$ is the derivative of $u$. Most other forms can be derived from these.
\subsection{Special Rules}
\par There are some more patterns outside of the traditional forms that the reader should know for Calculus. These have to do with composite and combined equations.
\paragraph{The Product Rule} Given two equations $f(x)$ and $g(x)$, the derivative of the function $h(x)=f(x)*g(x)$ is: $$h'(x)=f'(x)g(x)+g'(x)f(x)$$ For example, the derivative of $h(x)=x^2\,\sin(x)$ is: 
\begin{equation}
\begin{split}
h(x)&=f(x)*g(x) \\
h'(x)&=f'(x)g(x)+g'(x)f(x) \\
h'(x) &= 2x*\text{sin}(x)+\text{cos}(x)*x^2
\end{split}
\end{equation}
\vspace{.25in}\paragraph{The Quotient Rule} \textit{Avoid this at all costs.} The quotient rule is helpful but it is very, needlessly complex. For some function $h(x)$ that is the quotient of two functions $f(x)$ and $g(x)$: $h(x)=\displaystyle\frac{f(x)}{g(x)}$$$h'(x)=\frac{g(x)f'(x)-f(x)g'(x)}{(f(x))^2}$$ An easy way to remember this god-awful formula that was taught to me is a little song to the tune of \textit{Low Rider} by War:
\begin{displayquote}
	Low...d...High...minus High d Low...all...o-ver...the square of what's below 
\end{displayquote}
\begin{displayquote}
	(All...my...friends...know the low rider...the...low...rid-er...is a little higher)
\end{displayquote}
All jokes aside, this is one of the worst things in Calculus. Avoid it at all costs. For example, you could rewrite $\displaystyle\frac{3-x}{x^2}$ as $3x^{-2}-x^{-1}$, thereby eliminating all need for the quotient rule.
\vspace{.25in}\paragraph{The Chain Rule} The name is confusing. No, this rule doesn't have anything to do chains in the normal sense. Instead, it gives a form for a composite function $h(x)=f(g(x))$. This is required for functions like sin($x^2$) where sin$(x)=f(x)$ and $x^2=g(x)$. The chain rule says the derivative of $h(x)=f(g(x))$ is:$$h'(x)=f'(g(x))*g'(x)$$In our previous example, the derivative of sin$(x^2)$ is
\begin{equation}
\begin{split}
h(x) &= f(g(x)) \\
h'(x) &= f'(g(x))*g'(x) \\
h'(x) &= \text{cos}(x^2)*2x \\
h'(x) &= 2x*\text{cos}(x^2)
\end{split}
\end{equation}
\end{document}