\documentclass[../revisedmain.tex]{subfiles}
\begin{document}
	It has come up a bit before, but now it is time to formalize some theory. Because the derivative finds the rate of change for something at a certain point, it is useful in physics. For example, the derivative of a position function gives the rate of change of position of that point, or \textit{velocity}. Similarily, the change in velocity at a point is \textit{acceleration}. The change in acceleration at a point does get a special name in physics, too, but it is not used as much: \textit{the jerk}. Given $x(t)$ is position, $v(t)$ is velocity, and $a(t)$ is acceleration with respect to time, the following are true: $$x'(t)=v(t)$$$$x''(t)=v'(t)=a(t)$$ I did not include the jerk in here because it is so very uncommon.
\end{document}