\documentclass[../revisedmain.tex]{subfiles}
\begin{document}
	Taking the limit of a function with a fraction sometimes ends up with a confusing result like $\displaystyle\frac{\infty}{\infty}$ or $\displaystyle\frac{0}{0}$. Neither of these actually have a real value. L'h\^opital's rule is:$$\lim\frac{f(x)}{g(x)}=\lim\frac{f'(x)}{g'(x)}$$ \textbf{if and only if} $f(x)=g(x)$ and $f(x)$ is either $\pm\infty$ or 0. What this means is if, when taking the limit, you get one of these ``indeterminate forms'', you may take the derivative of the top and the bottom \textbf{separately} and then re-evaluate the limit. You may do this as many times as you need until you no longer reach an indeterminate form.
\end{document}