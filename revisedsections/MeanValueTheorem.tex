\documentclass[../revisedmain.tex]{subfiles}
\begin{document}
	The mean value theorem is relatively simple. If the reader would like to find a proof for the theorem, they are readily available on the internet.
	\begin{quotation}
		Given a closed interval [$a$,$b$], if a function $f(x)$ is always continuous, there exists a point $a\leq c\leq b$ such that the derivative at c is equal to the slope of the secant line through $a$ and $b$.
	\end{quotation}
	In other words, there exists some $c$ such that $$f'(c)=\frac{f(b)-f(a)}{b-a}$$You will see this on the AP test. The test will ask you which one of the given options is guaranteed by the mean value theorem. Simply find the slope of the secant line. Do note: this only works for continuous functions.
\end{document}