\documentclass[../revisedmain.tex]{subfiles}
\begin{document}
	Continuity does not guarantee that the derivative will always exist. Even though a function like
	\[
	f(x)=
	\begin{cases}
	x & x\leq 1\\
	x^2 & x>1 
	\end{cases}
	\]
	is continuous everywhere, its derivative is not. The derivative is a limit, so for it to exist, it must be the same on both sides. If the limit of the difference quotient (the derivative) is calculated at $x=1$, the results are:
	$$\lim_{x\to 1^-} f'(x) = \lim_{x\to 1^-} 1 = 1$$
	$$\lim_{x\to 1^+} f'(x)= \lim_{x\to 1^+} 2x = 2$$
	which are not equal. The derivative must not exist for all values of $x$. However, if the derivative of a function is continuous everywhere, that means that the function itself must be differentiable everywhere. Therefore, functions that do not result in a piecewise derivative and have a domain of $(-\infty,\infty)$ must be differentiable everywhere. Functions that are differentiable and expected to be calculated by Calculus students are::
	\begin{enumerate}
		\item Polynomials $(x^n+x^{n-1}+...)$
		\item Power functions $(ax^b)$
		\item Sin(x)
		\item Cos(x)
		\item Exponential functions $(ba^x)$
		\item Logarithmic functions
		\end{enumerate}
\end{document}