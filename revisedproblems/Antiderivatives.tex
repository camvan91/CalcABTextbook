\documentclass[../revisedmain.tex]{subfiles}
\begin{document}
So far we have worked with derivatives. Because the derivative is a function, it should have an inverse. It is obvious now that the derivative of $x^2$ is $2x$, so we should say the inverse of the derivative (or antiderivative) of $2x$ is $x^2$. This doesn't exactly work out as $x^2$, $x^2-1$, and $x^2+\pi$ all share $2x$ as their derivative. Instead, we say that the antiderivative of $2x$ is $x^2+C$ where $C$ is any constant. It should be noted that a function which is created by a derivative will have an infinite number of antiderivatives but only one derivative itself. Later in this section, you will read about ways to solve specific antiderivative problems.
\end{document}