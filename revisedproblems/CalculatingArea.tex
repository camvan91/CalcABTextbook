\documentclass[../revisedmain.tex]{subfiles}
\begin{document}
We have already established that the integral finds area. It can be used for more than just $f(x)$ graphs, though. For example, if we wanted to find the area of the graph of $f(t)=2\pi t$ (the circumfrence of a circle with radius $t$) for some radius $r$, the integral evaluates to:$$\int_{0}^{r} 2\pi t dt$$$$=2\pi\int_{0}^{r} t dt$$$$=\pi r^2$$Interesting, right? Anyways, you may ask why we start at 0. The answer is that it's convenient. We want to find the difference in area between some $t$ and, in this case, a circle with no area. That is why we use 0. It may be adventageous in other calculations to not start at 0, to find the area underneath the graph between some variable and some known value.
\end{document}