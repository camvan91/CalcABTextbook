\documentclass[../revisedmain.tex]{subfiles}
\begin{document}
To reduce the amount of error in our Riemann Sums, one could increase the total number of rectangles in the interval. 2,000 rectangles should be a closer approximation than 5 (generally). As we increase the number of rectangles, the width is reduced and the rectangles become closer to the graph. One way to minimize the total error is to maximize the number of rectangles. We can't directly compute $\infty$ rectangles but we can take the limit at $\infty$ subintervals. Turning the additions of the heights into a summation, we can express this as:$$\lim_{n\to\infty}\frac{b-a}{n}\sum_{i=1}^{n} f(a+i\frac{b-a}{n}) $$ Because we are dividing the interval [$a$,$b$] into $n$ subintervals and multiplying the height (function at each point) by the width (which is constant for all rectangles) and adding all of the rectangles together. The function $f(x)$ is evaluated at the first point $x=$ starting point + (how many rectangles we have already counted) * width, or $x=a+i\displaystyle\frac{b-a}{n}$. We end up needing to find this sum quite often, and as it is bulky to read, we use shorthand. We write a fancy \textit{s} that stands for \textit{sum} like this: $\int$ and place our starting and ending points on the \textit{s} like this: $\displaystyle\int_{a}^{b}$. We also abbreviate the width $\displaystyle\frac{b-a}{n}$ as $dx$ usually. It is \textit{d-some variable} always, and the variable we are moving along is usually $x$ because we are dividing up the $x-$axis. We can write our sum then as height * width or: $$\int_{a}^{b} f(x)*dx$$which is more commonly written as $$\int_{a}^{b}f(x)dx$$ We call this \textbf{the integral}. We read the last equation as \textit{the definite integral from a to b of f(x) dx}. The reason why we choose $dx$ to represent our width is because $dx$ is the infintesimal change in $x$, similar to the infintesimal change in x of the derivative. Because the derivative uses $dx$ to represent this change, we use the same notation to represent the change in our integral.
\end{document}