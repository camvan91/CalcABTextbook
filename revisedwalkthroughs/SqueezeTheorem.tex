\documentclass[../revisedMain.tex]{subfiles}
\begin{document}
	\newpage
\paragraph{Example 1:} The squeeze theorem is uncommon and only applies (to Calculus) in a small number of situations. As a result, we will only prove the previous problem and then move on. For the squeeze theorem to work, we need to show that two \textit{intersecting} graphs will be the maximum and minimum boundaries for a third function. In this case:
\begin{equation}
	\begin{split}
	f(x)&\le g(x)\\ 
	-x^2&\le x^2*\sin\left(\frac{4}{x}\right)\\
	-1&\le\sin\left(\frac{4}{x}\right)\\
	\end{split}
\end{equation}
That statement is true. The smallest value that \(\sin(t)\) for some \textit{t} can be is -1. Going the other way:
 \begin{equation}
 \begin{split}
 h(x)&\ge g(x)\\ 
 x^2&\ge x^2*\sin\left(\frac{4}{x}\right)\\
 1&\ge\sin\left(\frac{4}{x}\right)\\
 \end{split}
 \end{equation}
Again, a true statement. Because \(f(x)\) and \(h(x)\) intersect at \(x=0\), \(g(x)\) must be between \(f(0)=0\) and \(h(0)=0\) where the only solution is \(\lim\limits_{x\to 0}g(x)=0\).\\
\end{document}