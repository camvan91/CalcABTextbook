\documentclass[../revisedmain.tex]{subfiles}
\begin{document}
Knowing the integral is the antiderivative, we can solve differential equations now. All one has to do is separate the variables so all $y$'s and $dy$ are on one side and all $x$'s and $dx$ are on the other of the equality. For example, we can calculate the classic example of the slope of a graph being dependent upon the $y-$coordinate:$$\frac{dy}{dx}=yk$$$$\frac{1}{y}dy=k*dx$$$$\int\frac{1}{y}dy=k\int dx$$$$\ln(y)=kx+C$$$$y=e^{kx+C}$$$$y=e^{kx}*e^C$$$$y=e^{kx}*P$$$$y=Pe^{kx}$$ for some rate \textit{k} and some initial value (principal) $P$. $P$ is the intial value because when $x=0$, $y=Pe^{0}=P$.
\end{document}