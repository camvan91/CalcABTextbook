\documentclass[../revisedmain.tex]{subfiles}
\begin{document}
\paragraph{Example 1:}Estimate the value of $f(8.125)$ for $f(x)=-2\sqrt[3]{x^2}$
\begin{equation}
	\begin{split}
	f(x)&=-2\sqrt[3]{x^2}\\
	f(x)&=-2x^{\frac{2}{3}}\\
	\frac{df}{dx}&=-2x^{-\frac{1}{3}}\left(\frac{2}{3}\right)\\
	\frac{df}{dx}&=-\frac{4}{3}*\frac{1}{\sqrt[3]{x}}\\
	\frac{df}{dx}(8)&=-\frac{4}{3}*\frac{1}{2}\\
	\frac{df}{dx}(8)&=-\frac{2}{3}\\
	\end{split}
\end{equation}Now that we know the derivative at 8, we can plug that into a point-slope linear equation:
\begin{gather*}
	y-y_0=m(x-x_0)\\
	y-f(8)=-\frac{2}{3}(x-8)\\
	y+8=-\frac{2}{3}(x-8)\\
\end{gather*}We can now use that to estimate $f(8.125)$
\begin{equation}
	\begin{split}
	y+8&=-\frac{2}{3}(8.125-8)\\
	y&=-\frac{2}{3}*\frac{1}{8}-8\\
	y&=-\frac{1}{12}-8\\
	y&=-8\,\frac{1}{12}\\
	\end{split}
\end{equation} A good approximation for $f(8.125)$ is therefore $-8\,\displaystyle\frac{1}{12}$.\\
\end{document}