\documentclass[../revisedmain.tex]{subfiles}
\begin{document}
\paragraph{Example 1:}The velocity of a particle moving along the $x$-axis is $v(t)=t^2-10t+30$. What is the change in position from time $t=0$ to $t=7$?\\\newline\par The AP test has multiple problems where a particle is moving along the $x$-axis dependent on time $t$. Don't think of this as a 3-dimensional problem even though it seems like it. This is \textit{not} how you should think about it:\\\vspace{.125in}
\begin{center}
\resizebox{3.5in}{!}{
\begin{tikzpicture}
	\begin{axis}
	[
	view={45}{30},
	axis lines = left,
	xlabel=$x$,
	ylabel=$y$,
	zlabel=$t$,
	]
	\addplot3
	[
	domain=0:8,
	samples=60,
	samples y=0,
	color=red,
	]
	({x^2-10*x+30},
	{x},
	{.125*x^2});
	\end{axis}	
\end{tikzpicture}
}
\end{center}
Because they're only asking for the change in the $x$-coordinate with relation to time $t$. This is how you should approach the problem:
\begin{center}
	\resizebox{3.5in}{!}{
	\begin{tikzpicture}
	\begin{axis}
	[
	axis lines=left,
	xlabel=$t$,
	ylabel=$x$,
	]
		\addplot
		[
		domain=0:8,
		samples=60,
		color=red,
		]
		{x^2-10*x+30};
	\end{axis}
	\end{tikzpicture}
}
\end{center}
Now that the confusing wording is out of the way, the problem is pretty simple. All we need to do is find the net change in position given the velocity:
\begin{gather*}
	v(t)=t^2-10t+30\\
	\int_{0}^{7}v(t)\,dt=\int_{0}^{7}t^2-10t+30\,dt\\
	\int_{0}^{7}v(t)\,dt=\frac{t^3}{3}-5t^2+30t\,\Big|_{0}^{7}\\
	\int_{0}^{7}v(t)\,dt=\frac{343}{3}-245+210\\
	79\frac{1}{3}
\end{gather*}
\end{document}