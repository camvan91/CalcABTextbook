\documentclass[../revisedmain.tex]{subfiles}
\begin{document}
	This is perhaps the most classic Calculus derivative problem. When you are riding in a car, the speedometer tells you how fast you are going at any point in time. This is your instantaneous rate of change, how fast you are moving at one point in time. You could be travelling at 40 miles per hour but then have to stop at a stoplight for a little bit. Your average speed would be less than 40 because of that time you stopped, but you know that at that certain time you measured it, you were travelling 40 miles per hour. Problems will ask you, given a function, how fast is something changing at an exact point in time. All it is asking is for the slope of the tangent line (derivative) and not the average slope at that point. Simply calculate the derivative.
\end{document}