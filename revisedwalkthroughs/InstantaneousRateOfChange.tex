\documentclass[../revisedmain.tex]{subfiles}
\begin{document}
\paragraph{Example 1:}Find the instantaneous rate of change at $t=3$ and the average rate of change over the interval $2\le t\le6$ for the function $f(t)=\displaystyle\frac{t}{t^2-3}$\\
\begin{equation}
	\begin{split}
	f(t)&=\frac{t}{t^2-3}\\
	f(t)&=t*(t^2-3)^{-1}\\
	\frac{df}{dt}&=(t^2-3)^{-1}+t*-1(t^2-3)^{-2}(2t)\\
	\frac{df}{dt}&=\frac{1}{t^2-3}-\frac{2t^2}{(t^2-3)^2}\\
	\frac{df}{dt}&=\frac{t^2-3}{(t^2-3)^2}-\frac{2t^2}{(t^2-3)^2}\\
	\frac{df}{dt}&=-\frac{t^2+3}{(t^2-3)^2}\\
	\frac{df}{dt}(3)&=-\frac{12}{(6)^2}\\
	\frac{df}{dt}(3)&=-\frac{1}{3}\\
	\end{split}
\end{equation}So the instantaneous rate of change is $-\frac{1}{3}$ when $t=3$. Next, the average rate of change:
\begin{gather*}
\frac{f(6)-f(2)}{6-2}\\[.5em]
\frac{\displaystyle\frac{6}{33}-\displaystyle\frac{2}{1}}{4}\\[.5em]
\frac{\displaystyle\frac{2}{11}-\displaystyle\frac{22}{11}}{4}\\[.5em]
\frac{-\displaystyle\frac{20}{11}}{4}\\[.5em]
-\frac{20}{11}*\frac{1}{4}\\[.5em]
-\frac{5}{11}\\
\end{gather*}Remember, \textit{instantaneous rate of change} means the rate of change \textit{at a single point}. The derivative finds the rate of change at a point if the function is differentiable at that point. It's just a complicated way of saying ``take the derivative".
\end{document}