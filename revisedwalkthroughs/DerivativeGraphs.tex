\documentclass[../revisedMain.tex]{subfiles}
\begin{document}
	Functions and their derivatives are connected. If the derivative is positive, the function is increasing. If the derivative is negative, the function is decreasing. If the second derivative is positive, the function is accelerating in the positive direction. One of the many things a Calculus student is expected to do is identify the relation between the graphs of the derivatives of a function and the function itself. Let $f(x)=(x-1)^3$: 
	\begin{equation}
		\begin{split}
			f(x)&=(x^2-2x+1)(x-1)\\
			f(x)&=x^3-2x^2+x-x^2+2x-1\\
			f(x)&=x^3-3x^2+3x-1\\
			f'(x)&=3x^2-6x+3\\
			f''(x)&=6x-6
		\end{split}
	\end{equation}
	Graphing the functions, we get:
			\begin{center}
				\begin{tikzpicture}
				\begin{groupplot}[group style={group size=2 by 2}, height=6cm, width=6cm, axis lines=middle]
				\nextgroupplot[title=$f(x)$]
				\addplot [blue,domain=0:2]{(x-1)*(x-1)*(x-1)};
				\nextgroupplot[title=$f'(x)$]
				\addplot [green,domain=0:2]{3*x^2-6*x+3};
				\nextgroupplot[title=$f''(x)$]
				\addplot [red,domain=0:2]{6*x-6};
				\end{groupplot}
				\end{tikzpicture}
			\end{center}
			$f'(x)$ has a zero at $x=1$ but because it doesn't change signs, $f(x)$ cannot have a local maximum or minimum there. Because $f'(x)$ is always positive, the function is always increasing. $f''(x)$ is negative before $x=1$ and positive after, so at that point, $f(x)$ switches from concave down to concave up. This sums up the relationship between derivatives and their graphs.
\end{document}