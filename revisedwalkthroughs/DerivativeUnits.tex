\documentclass[../revisedmain.tex]{subfiles}
\begin{document}
	Functions can be used to model real-world phenomena and find use in everything from quantum physics to music\footnote{Just ask Bach...}. However, in the real world, there are units of measure that are significant and go hand-in-hand with values. Think of \textit{miles}: a unit of distance, and \textit{hours}: a unit of time. We can combine these together to show a relationship between the two with \textit{miles per hour}: a  unit of speed. Because derivatives give rates of change, the units no longer stay the same. For example, if we wanted to find the rate of change of speed, we would look for how the speed changes with relation to time. Our units then would be $\displaystyle\frac{miles}{hour}\div hour$ or miles per hour per hour. We rewrite this as $\displaystyle\frac{miles}{hour^2}$. This gives us the rate of change of speed, or acceleration. If we wanted to find how the area of a circle in square meters changes when we change the radius in meters, the units would be $\displaystyle\frac{meters^2}{meters} = meters$.
\end{document}