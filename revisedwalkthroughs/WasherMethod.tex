\documentclass[../revisedmain.tex]{subfiles}
\begin{document}
\paragraph{Example 1:}Find the volume of the solid bounded by $y=x^2$, $y=\sin\left(\pi x^2\right)+x$, $x$=0, and $x$=1 revolved around the $y$-axis.\newline
The only way to calculate this is by finding the difference between $x^2$ and $\sin\left(\pi x^2\right)+x$, so we must use the shell method. Remember, the shell of a cylinder is $2*\pi*r*height*width$. $r$ is the distance away from the axis of revolution, which in this case is $x-0=x$. $height$ will be the height of the rectangle, or $\sin\left(\pi x^2\right)+x-x^2$. $width$ is going to be infinitely small, or $dx$. That leaves us with the volume being:\[dV=2\pi*x*\left(\sin\left(\pi x^2\right)+x-x^2\right)*dx\]which simplifies down to\[dV=2\pi \left(x\sin\left(\pi x^2\right)+x^2-x^3\right)dx\]Now, we will add up all of these volumes between $x=0$ and $x=1$:
\begin{gather*}
	V=\int_{0}^{1}2\pi \left(x\sin(\pi x^2)+x^2-x^3\right)dx\\
	V=2\pi\left(\int_{0}^{1}x\sin(\pi x^2)dx+\int_{0}^{1}x^2dx-\int_{0}^{1}x^3dx\right)\\
	u=x^2\\
	du=2x\,dx\\
	x=0\mapsto u=0\\
	x=1\mapsto u=1\\
	V=2\pi\left(.5\int_{0}^{1}\sin(\pi u)du+\int_{0}^{1}x^2dx-\int_{0}^{1}x^3dx\right)\\
	V=2\pi\left(\frac{1}{2\pi}(-\cos(\pi*1)+\cos(\pi*0)\right)+\frac{1}{3}-\frac{0}{3}-\frac{1}{4}+\frac{0}{4}\\
	V=2\pi\left(\frac{1}{2\pi}(1+1)+\frac{1}{3}-\frac{1}{4}\right)\\
	V=2+\frac{\pi}{6}\\
\end{gather*}
\paragraph{Example 2:}Find the volume of the solid bounded by $y=x^2$, $y=4$, and $x=0$ revolved around the y-$axis$.\\It's probably best to approach this problem using the washer method. First, we have to find everything in terms of $y$:
\begin{gather*}
	y=x^2\\
	\sqrt{y}=x\\
\end{gather*}
Now for the boundary points $x=0$ and $x=1$
\begin{gather*}
	x=0\mapsto y=0\\
	x=2\mapsto y=4\\
\end{gather*}
Here's what we have so far: the boundaries are $y=0$ and $y=4$, the function is $x=\sqrt{y}$, and we are revolving around $x=0$. Our infinitesimal volume is $\pi*r^2*height$ which is:
\[dV=\pi\left(\sqrt{y}\right)^2*dy\]
Therefore, our total sum of all of the volumes is:
\begin{gather*}
V=\int_{0}^{4}\pi*(x)^2*dy\\
V=\pi\int_{0}^{4}(\sqrt{y})^2dy\\
V=\pi\int_{0}^{4}y\, dy\\
V=\pi\left(.5(4)^2-.5(0)^2\right)\\
V=8\pi\\
\end{gather*}
\end{document}