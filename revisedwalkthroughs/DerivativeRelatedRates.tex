\documentclass[../revisedmain.tex]{subfiles}
\begin{document}
	Derivatives find the rate of change of something with respect to something else. Two or more derivatives can be combined, as a result, to find how one value affects another. For example, if we know the the radius of a circle at any given time is $\ln(t)$, we know the rate of change musb be $\displaystyle\frac{1}{t}$, which we can rewrite as $\displaystyle\frac{dr}{dt}=\displaystyle\frac{1}{t}$. If we wanted to find how the area of a circle changes with respect to the radius, all we need to do is take the derivative. $$A=\pi r^2$$$$\frac{dA}{dt}=2\pi r\frac{dr}{dt}$$ substituting for $r$ and $\displaystyle\frac{dr}{dt}$, we get $$\frac{dA}{dt}=2 \pi \ln(t) * \frac{1}{t}$$ which means that for any given time $t$, the rate of change of the area $A$ must be $$\frac{2\ln(t)}{t}$$ Is it not super cool that the derivative of the area of a circle is the circumfrence?
\end{document}