\documentclass[../revisedmain.tex]{subfiles}
\begin{document}
\paragraph{Example 1:} Find when the function \(f(x)=1/4x^4+x^3-1/2x^2-3x+17\) is: Concave up increasing, concave down increasing, concave up decreasing, and concave down decreasing.
\begin{equation}
	\begin{split}
	f(x)&=1/4x^4+x^3-1/2x^2-3x+17\\
	f'(x)&=x^3+3x^2-x-3\\
	f''(x)&=3x^2+6x-1 \\
	\end{split}
\end{equation}First, we will find when the first derivative is positive or negative to determine where \(f(x)\) is increasing or decreasing.
\begin{equation}
	\begin{split}
	f'(x)&=x^3+3x^2-x-3\\
	f'(x)&=x^3-x+3x^2-3\\
	f'(x)&=x(x^2-1)+3(x^2-1)\\
	f'(x)&=(x+3)(x^2-1)\\
	f'(x)&=(x+3)(x-1)(x+1)\\
	0&=(x+3)(x-1)(x+1)\\
	x&=-3,\pm 1 \\
	\end{split}
\end{equation}
\begin{center}\begin{tabular}{|c|c|c|c|}
	\hline
	\((-\infty ,-3)\) & \((-3,-1)\) &\((-1,1)\) & \((1,\infty)\)\\
	\hline
	-&+&-&+\\
	\hline
\end{tabular}\end{center}
So \(f(x)\) is increasing over \((-3,-1)\) and \((1,\infty)\), and decreasing over \((-\infty,-3)\) and \((-1,1)\). Now, we will find where the function is concave up and down:
\begin{equation}
	\begin{split}
	f''(x)&=3x^2+6x-1\\
	0&=3x^2+6x-1\\
	x&=\frac{-6\pm \sqrt{6^2-4*3*-1}}{2*3}\\
	x&=\frac{-6\pm\sqrt{48}}{6}\\
	x&=\frac{-3\pm 2\sqrt{3}}{3}\\
	x&\approx -2.155, 0.155 \\
	\end{split}
\end{equation}
\begin{center}\begin{tabular}{|c|c|c|}
	\hline
	\((-\infty,-2.155)\) & \((-2.155,-0.155)\) & \((-0.155,\infty)\) \\
	\hline
	+&-&+\\
	\hline
\end{tabular}\end{center}So \(f(x)\) is concave up over \((-\infty,-2.155)\) and \((-0.155,\infty)\), and concave down over \((-2.155,-0.155)\).
\paragraph{Example 2:}Find where \(\sin(x-\frac{\pi}{2})\) is concave up and down:
\begin{equation}
	\begin{split}
	f(x)&=\sin(x-\frac{\pi}{2})\\
	f'(x)&=\cos(x-\frac{\pi}{2})\\
	f''(x)&=-\sin(x-\frac{\pi}{2})\\
	\end{split}
\end{equation}
	\begin{tikzpicture}
	\begin{axis}[axis lines=middle,domain=0:6.28]
	\addplot[color=red,domain=0:6.28,samples=50]{sin(180*x/pi-90)};
	\end{axis}
\end{tikzpicture}\\
\begin{center}
	\small{\(\sin(x-1)\)}
\end{center}
\par Interestingly, the concavity is just the inverse of the function itself. sin\((x-\frac{\pi}{2})\) is concave up when it is negative, written out mathematically: \((-\frac{\pi}{2},\frac{\pi}{2})+n*2\pi\), where \(n\) is any integer.\\\\
\end{document}